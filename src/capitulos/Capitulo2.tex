% Capítulo 2
\chapter{A caminhada de Arash}

A caminhada de robôs com pernas vem sendo desenvolvida desde da década de 70, quando Kato e Vukobratovic praticamente iniciaram este campo de estudo \cite{kajita2008}.

E no escopo do desenvolvimento de um \textit{walking gait} é possível duas estratégia de equilíbrio: estática ou dinâmica. Cada estratégia tem suas vantagens e desvantagens. Todavia, o equilíbrio dinâmico vem sendo mais aceito dentre os pesquisadores.
	
O equilíbrio estático mantém o centro de massa, ou \abrv[CoM -- Centro de massa, do inglês \textit{Center of Mass}]{CoM -- do inglês \textit{Center of Mass} --} sempre sob a superfície de apoio do robô. Uma desvantagem desse modelo é que as velocidades alcançadas são bem lentas, tendo em vista que o movimento produzido para a caminhada não deve gerar inercia suficiente para deslocar o CoM fora da área esperada. Geralmente observa-se pés maiores em robôs que adotam este tipo de caminhada \cite{miller1994}.

Ainda, segundo Miller, o equilíbrio dinâmico, considera a dinâmica do movimento da caminhada, onde o \abrv[ZMP -- do inglês \textit{zero moment point}]{ZMP, do inglês \textit{zero moment point},} sempre fica sob a superfície de apoio. Desta forma, habilita-se o CoM a movimentar-se para fora do ponto de apoio, mantendo o robô em uma espécie de ``queda controlada''. Assim, ela provê maiores velocidades de caminhada com maior eficiência. O ZMP é o ponto de apoio com o chão onde nenhum momento é criado no eixo horizontal \cite{kajita2008}.

Durante qualquer tarefa envolvendo locomoção é necessária uma maneira de controlar o \textit{walking gait} de forma a direcionar o robô ao destino escolhido. Uma das abordagens utilizadas são movimentos pré-definidos que executam ações específicas. Por exemplo, dar um passo a frente, um passo lateral à esquerda, girar 30\degree à direita. Assim, essas ações são sequenciadas de forma a levar o robô até seu destino. A desvantagem deste método é cada ação deve ser iniciada e finalizada antes que a próxima seja acionada e isto diminui a agilidade do robô.

Uma segunda forma de controle é a chamada caminhada omnidirecional que consiste em controlar o robô por velocidades em diferentes eixos. Essas mudanças podem ser realizadas com o robô em movimento, sem a necessidade de encadeamento de ações. Além disso, existe a possibilidade de combinar movimentos em diferentes eixos ao mesmo tempo criando novas possibilidades de locomoção.

Neste trabalho, a abordagem de Karimi \textit{et al} é usada como base para o desenvolvimento do novo \textit{walking gait}. Eles descrevem seu método como uma caminhada omnidirecional com equilíbrio dinâmico.