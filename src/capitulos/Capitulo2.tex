% Capítulo 2
\chapter{Capítulo 2}
\todo{Definir título}

A caminhada de robôs com pernas vem sendo desenvolvida desde da década de 70, quando Kato e Vukobratovic praticamente iniciaram este campo de estudo \cite{kajita2008}.

\begin{draft}
	\todo{Reescrever esta desgraça}
	\todo{Introduzir os conceitos de equilíbrio estático e dinâmico.}No desenvolvimento de qualquer \textit{walking gait}, é necessário a estratégia de equilíbrio entre estática ou dinâmica. Cada estratégia tem suas vantagens e desvantagens. Todavia, o equilíbrio dinâmico vem sendo mais aceito dentre os pesquisadores.
\end{draft}

O equilíbrio estático mantém o centro de massa, ou \abrv[CoM -- Centro de massa, do inglês \textit{Center of Mass}]{CoM -- do inglês \textit{Center of Mass} --} sempre sob a superfície de apoio do robô. Uma enorme desvantagem é que as velocidades alcançadas são bem lentas, tendo em vista que o movimento produzido para a caminhada não deve gerar inercia suficiente para deslocar o CoM fora da área esperada. Geralmente observa-se pés maiores em robôs que adotam este tipo de caminhada \cite{miller1994}.

Ainda segundo Miller, o equilíbrio dinâmico, considera a dinâmica do movimento da caminhada, as inércias das partes que movem-se. Com isto, ela habilita o CoM movimentar-se para fora do ponto de apoio, mantendo o robô em uma espécie de ``queda controlada''. Assim, ela habilita maiores velocidades de caminhada com maior eficiência.

\todo{Ligar parágrafos}

Durante qualquer tarefa envolvendo locomoção, é necessário uma maneira de controlar o \textit{walking gait} de forma a direcionar o robô ao destino escolhido. Uma das formas utilizadas são movimentos pré-definidos que executam ações específicas. Ações como, um passo a frente, um passo lateral à esquerda (e à direta), girar 30\deg{} à direita (e à esquerda), entre outros em uma sequência específica levaram o robô até o seu objetivo. Outra forma de controle é a chamada caminhada omnidirecional.

A caminhada omnidirecional 

\begin{guide}
	Falar sobre as tecnologias atuais de \textit{walking gait}.
\end{guide}

\begin{guide}
	Falar sobre a abordagem tomada para este trabalho.
\end{guide}

Este é o primeiro capítulo da parte central do trabalho, isto é, o desenvolvimento, a parte mais extensa de todo o trabalho. Geralmente o desenvolvimento é dividido em capítulos, cada um com subseções e subseções, cujo tamanho e número de divisões variam em função da natureza do conteúdo do trabalho.

Em geral, a parte de desenvolvimento é subdividida em quatro subpartes:

\begin{itemize}
   \item \textit{contextualização ou definição do problema} -- consiste em
   descrever a situação ou o contexto geral referente ao assunto em questão,
   devem constar informações atualizadas visando a proporcionar maior
   consistência ao trabalho;
   \item \textit{referencial ou embasamento teórico} -- texto no qual se deve
   apresentar os aspectos teóricos, isto é, os conceitos utilizados e a
   definição dos mesmos; nesta parte faz-se a revisão de literatura sobre o
   assunto, resumindo-se os resultados de estudos feitos por outros autores,
   cujas obras citadas e consultadas devem constar nas referências;
   \item \textit{metodologia do trabalho ou procedimentos metodológicos} -- deve
   constar o instrumental, os métodos e as técnicas aplicados para a elaboração
   do trabalho;
   \item \textit{resultados} -- devem ser apresentados, de forma objetiva,
   precisa e clara, tanto os resultados positivos quanto os negativos que foram
   obtidos com o desenvolvimento do trabalho, sendo feita uma discussão que
   consiste na avaliação circunstanciada, na qual se estabelecem relações,
   deduções e generalizações.
\end{itemize}

É recomendável que o número total de páginas referente à parte de
 desenvolvimento não ultrapasse 60 (sessenta) páginas.