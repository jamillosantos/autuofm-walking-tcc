% Introdução
\chapter{Introdução}

\begin{guide}
	\begin{itemize}
		\item Contexto
		\item Motivação
		\item O problema
		\item Soluções similares
		\item Solução proposta
	\end{itemize}
\end{guide}

\begin{guide}
	Falar sobre a robótica em amplo contexto e que somos cercados por robôs especialistas.

	Falar o porquê de robótica humanoide. Mundo de humanos. Ferramentas projetadas
	para humanos. Qual o \textit{design} adaptaria-se melhor?

	Falar das dificuldades na robótica humanoide e fazer uma contextualização
	da caminhada citando alguns trabalhos recentes.
\end{guide}

\begin{guide}
	Falar sobre o consórcio entre as universidades da AUT e UofM e da sua
	participação na RoboCup. Assim concluir o porquê utilizar a sua abordagem
	de \textit{walking gait}.
\end{guide}

\begin{guide}
	Falar o porquê a o \textit{walking gait} do AUT-UofM precisa de melhoria.
\end{guide}

\begin{guide}
	Falar sobre a solução.
\end{guide}


Quando pensamos em robótica, imaginamos um cenário futurista cheio de tecnologia
com robôs super inteligentes nos substituindo em diversos aspectos. Porém,
nos dias de hoje já estamos cercados de robôs extremamente especialistas e
eficientes. Máquinas que automatizam nossas tarefas mais corriqueiras do dia
a dia, como veículos de transporte, aspiradores de pó inteligentes e até
mesmo liquidificadores. Assim, robôs são muito tão especializados que apenas
servem para um propósito.

A caminhada de robôs humanoides é um componente essencial para a realização 
a maioria das tarefas relacionadas a este tipo de robô. Afinal locomover-se
está no centro de muitas tarefas e uma caminhada estável provê uma fundação
firme para a utilização deste componente no desenvolvimento de outras tarefas
muito mais complexas.


Nas edições 2015 e 2016 da RoboCup, realizada em Hefei (China) e Leipzig (Alemanhã),
o sistema que controlou a caminhada de \textit{Arash}; um robô humanóide de
105 centímetros de altura que tirou o 3º lugar no futebol de robô (2015 e 2016),
2º e 1º lugar no desafio técnico em 2015 e 2016, respectivamente; foi desenvolvido
utilizandoplaca microcontroladora \textit{OpenCM 9.04}, compatível com Arduino,
e batizado de \textit{AUT-UofM-Walk-Engine}. Porém, para configurar este componente
é necessário fazer alterações diretas no código, compilá-lo e enviá-lo à \textit{OpenCM9}.
Fazer isto durante a competição, um ambiente com mudanças constantes e com uma forte
restrição de tempo, torna difícil a vida da equipe.

\section{Objetivos}

Nesta seção são definidos os objetivos gerais e específicos do trabalho.

\subsection{Objetivos Gerais}

\begin{itemize}
	\item Criar componente \textit{walking gait}, controlado em runtime via JSON over UDP, com ferramenta de configuração simplificada.
\end{itemize}

\subsection{Objetivos Específicos} 

\begin{itemize}
	\item Melhorar o código atual (legibilidade e arquitetura);
	\item Diminuir a curva de aprendizado, tornando o código extensível;
	\item Integrar componente com o ROS (Robot Operating System) e Gazebo (simulação);
	\item Diminuindo o tempo, necessidade de conhecimento sobre o código para efetuar a configuração.
\end{itemize}

\section{Metodologia}

Na metodologia é descrito o método de investigação e pesquisa para o
desenvolvimento e implementação do trabalho que está sendo proposto.

\section{Organização do trabalho}

Nesta seção deve ser apresentado como está organizado o trabalho, sendo
descrito, portanto, do que trata cada capítulo.
