% Resumo em l�ngua estrangeira (em ingl�s Abstract, em espanhol Resumen, em franc�s R�sum�)
\begin{center}
	{\Large{\textbf{Título do trabalho (em língua estrangeira)}}}
\end{center}

\vspace{1cm}

\begin{flushright}
	Author: \writeauthor\\
	Supervisor: \writeteacheren
\end{flushright}

\vspace{1cm}

\begin{center}
	\Large{\textsc{\textbf{Abstract}}}
\end{center}

The robotics field is increasingly advanced and accessible. Whether it is a small standalone vacuum cleaner robot or the sophisticated robots which automate car manufacturing, increasing production and lowering your production cost. However, in their very varied forms, they do not always suit the environments designed for humans. Therefore, research with humanoid robots has significant importance for our days. In this work the Arash robot was used, a humanoid robot 1m tall and 7.5 kg of weight. This robot controlled using software implemented in a structured paradigm. The goal of this work is to reimplement one part of this control, the walking gait, in an object oriented language. The new Walking gait has a configuration scheme based on JSON objects saved into files. In this way, it is possible to initialize the component with several different parameters. It is also possible to version the configuration files so that changes are kept traceable. To demonstrate the feasibility of this approach a set of checks were performed and showed the effectiveness of the new walking gait.

\textit{Keywords}: Robot, Humanoid, Control, Walking Gait, Walking.