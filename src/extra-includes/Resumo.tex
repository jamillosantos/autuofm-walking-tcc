% Resumo
\begin{center}
	{\Large{\textbf{Título do trabalho}}}
\end{center}

\vspace{1cm}

\begin{flushright}
	Autor: Jamillo Santos\\
	Orientador(a): Titulação e nome do(a) orientador(a)
\end{flushright}

\vspace{1cm}

\begin{center}
	\Large{\textsc{\textbf{Resumo}}}
\end{center}


\noindent Este trabalho apresenta a implementação do \textit{Walking Gait} utilizado pelo time AUT-UofM nas Robocup 2015 e 2016 em C++. A versão anterior utilizava uma placa microcontroladora OpenCM9, baseada em Arduino, que impunham algumas dificuldades de manutenção, extensão e testes (unitários e de integração).

Um dos principais componentes do \textit{Walking Gait} é a compensação de equilíbrio. Para isto, o componente usa uma \textit{IMU} para capturar as informações necessárias para detectar perturbações externas. Para a normalização dos dados capturados da \textit{IMU}, um filtro de Kalman é utilizado. Para a correta validação deste componente, um \textit{MOCK} do sensor \textit{IMU} do celular.

Para o configuração dos parâmetros para o controle e customização da caminhada, foi desenvolvido uma interface web responsiva. Assim, a configuração poderá ser realizada de qualquer plataforma.

Por fim, a validação do sistema de dá em duas etapas. A primeira se dá através de uma simulação no Gazebo/ROS. Posteriormente, o sistema é testado em \textit{ARASH}, um robô humanóide de 1,05m.

\noindent O resumo deve apresentar de forma concisa os pontos relevantes de um
texto, fornecendo uma visão rápida e clara do conteúdo e das conclusões do
trabalho. O texto, redigido na forma impessoal do verbo, é constituído de uma
sequência de frases concisas e objetivas e não de uma simples enumeração de
tópicos, não ultrapassando 500 palavras, seguido, logo abaixo, das palavras
representativas do conteúdo do trabalho, isto é, palavras-chave e/ou
descritores. Por fim, deve-se evitar, na redação do resumo, o uso de parágrafos
(em geral resumos são escritos em parágrafo único), bem como de fórmulas,
diagramas e símbolos, optando-se, quando necessário, pela transcrição na forma
extensa, além de não incluir citações bibliográficas.

\noindent\textit{Palavras-chave}: Robótica, Humanoide, Walking Gait?, Arduino, C++.