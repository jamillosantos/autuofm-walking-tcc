% Resumo
\begin{center}
	{\Large{\textbf{\writedocumenttitle}}}
\end{center}

\vspace{1cm}

\begin{flushright}
	Autor: \writeauthor \\
	Orientador(a): \writeteacher
\end{flushright}

\vspace{1cm}

\begin{center}
	\Large{\textsc{\textbf{Resumo}}}
\end{center}

\noindent A robótica está cada vez mais avançada e acessível. Seja em um pequeno robô aspirador de pó autônomo ou nos sofisticados robôs que automatizam a fabricação de carros, aumentando a produção e diminuindo seu custo de produção. Entretanto, em suas formas muito variadas, nem sempre se adequam aos ambientes projetados para os humano. Por isso, a pesquisa com robôs humanoides tem significativa importância para nossos dias. Nesse trabalho foi utilizado o robô Arash, humanoide com 1m de altura e 7,5 kg de peso. Esse robô funciona a partir de controles em software, implementados em paradigma estruturado. O objetivo desse trabalho é reimplementar uma parte desse controle, o \textit{walking gait}, em linguagem orientada a objetos. O novo Walking gait possui um esquema de configurações baseado em objetos JSON salvos em arquivos. Desta forma, é possível inicializar o componente com diversos parâmetros diferentes. Também é possível o versionamento dos arquivos de configuração, para que alterações sejam mantidas de forma rastreável. Para demonstrar a viabilidade dessa abordagem um conjunto de verificações foram realizadas e mostrou a eficácia do novo \textit{walking gait}.
 
\noindent\textit{Palavras-chave}: Robô, Humanoide, Controle, Walking Gait, Caminhada.
